Our results are very close to those predicted by the model. It is important to note that the discrepant points are at the edge of the measured range, and we can also see from figure \ref{fig: Sets graph} that they do not follow the trend of the other points, being higher than the previous point in a descending function. Furthermore, these points are outside the uncertainty range by only small amounts, which, all things considered, is why we believe that despite these discrepancies with the model we can still consider that our results support the theoretical basis for this experiment.

We must also note that, even though we did our best to center the rail along the axis of the coil, we were clearly off center given that our data shows that there was some magnetic field strength perpendicular to the axis where we were taking our measurements, which would not be expected if we really were on the axis. Additionally, we must also mention that it did not occur to us, at the time of the experiment, to align the rail vertically which would also affect our results.

Additionally, any change in the background field, due, for example, to the use of nearby electronic devices, would also significantly influence the results obtained.

The number of data points also limits our ability to draw meaningful conclusions, it would have been interesting to take measurements at more positions to be able to draw stronger conclusions.

Moving forward, now that we have established a predictable magnetic field, it would be interesting to study magnetic flux and possibly induction. Since we can predict the change in magnetic flux experienced by an object with a known velocity, we could experiment with inductions, which could help validate our model on this subject.